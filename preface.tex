
%\chapter*{什么是数学}


\chapter*{什么是数学}
\addcontentsline{toc}{chapter}{什么是数学}

数学,作为人类思维的表达方式,反映了人们积极进取的意志、缜密周详的推理以及对完美境界的追求。它的
基本要素是:逻辑和直观、分析和构作、一般性和个别性。虽然不同的传统可以强调不同的侧面,然而正是这些
互相对立的力量的相互作用以及它们综合起来的努力才构成了数学科学的生命、用途和它的崇高价值。

毫无疑问,一些数学的发展在心里上都或多或少地是基于实际的。但是理论一旦在实际的需要中出现,就不可避免
地会使它自身获得发展的动力,并超越出直接实用的局限。这种从应用科学到理论科学的发展趋势,不仅常见于
古代历史中,而且在工程师和物理学家为近代数学不断作出的许多贡献中更是屡见不鲜。

有记载的数学起源于东方。大约在公元前两千年,巴比伦人就搜集了及其丰富的资料,这些资料今天看来应属于
初等代数的范围。至于数学作为现代意义的一门科学,则是迟至公元前五至公元前四世纪才在希腊出现的。东方和希腊
之间的接触不断增多(始于波斯帝国时期,至亚历山大远征时期则达到高峰),使希腊人得以熟悉巴比伦人在数学和天文学
方面的成就,数学很快就被加入到风行于希腊城邦的哲学讨论之中,因而希腊的思想家逐渐意识到,在连续、运动、无限大这些
概念中,以及在用已知单位去度量任意一个量的问题中,数学都存在着固有的极大困难。面对这个挑战,经过了一番不屈不挠的努力,
产生了欧多克斯(Eudoxus)的几何连续统理论,这个成果是唯一能和两千多年后的现代无理数理论相媲美的。数学中这种公理演绎的
趋向起源于欧多克斯时代,又在欧几里得(Euclid)的“原本”中得以成熟。

虽然希腊数学的理论化和公理化的倾向一直是它的一个重要特性,并且曾经产生过巨大的影响。但是,对这一点我们不能过分
强调,因为在古代数学中,应用以及同物理现实的联系恰恰起了同样重要的作用,而且那时候人们宁愿采用不像欧几里得那样
严密的表达方式。

由于较早地发现了与“不可公度”的量有关的这些困难,使希腊人没能发展早已为东方所掌握的数字计算的技术。相反,他们却迫使
自己钻进了纯粹公理几何的丛林之中。于是科学史上出现了一个奇怪的曲折。这或许意味着人类丧失了一个很好的时机。几乎两千年来,
希腊几何的传统力量推迟了必然会发生的数的概念和代数运算的进步,而它们后来构成了近代科学的基础。

经过了一段缓慢的准备,到十七世纪,随着解析几何与微积分的发展,数学和科学的革命也开始蓬勃发展起来。虽然希腊的几何学仍然占有
重要的地位,但是,希腊人关于公理体系和系统推演的思想在十七世纪和十八世纪不复出现。从一些清清楚楚的定义和没有矛盾的“明显”
公理出发,进行准确的逻辑推理,这对于数学科学的新的开拓者来说似乎是无关紧要的。通过毫无拘束的直观猜想和令人信服的推理,
再加上荒谬的神秘论以及对形式推理的超人力量的盲目相信,他们征服了一个蕴藏着无限财富的数学世界。但是后来,大发展引起的狂热
逐渐让位于一种自我控制的批判精神。到了十九世纪,由于数学本身需要巩固已有成果,而且人们也希望把它推向更高阶段时不致发生问题
(这是受到法国大革命的影响),就不得不回过头来重新审查这新的数学基础,特别是微积分及其赖以建立的极限概念。因此十九世纪不仅成为
一个新的发展时期,而且也以成功地返回到那种准确而严谨的证明为其特征。在这方面它甚至胜过了希腊科学的典范。于是,钟摆又一次向
纯粹性和抽象性的一侧摆去。目前我们似乎仍然处于这个时期。但是人们可以期望,在纯粹数学和具有活力的应用之间产生了这种不幸分离
(可能在批判性的审查时期,这是不可避免的)之后,随之而来的应是一个紧密结合的时代。这种重新获得的内在力量,更主要的是由于理解更加
明晰而达到认识上的极大简化,将使得今天有可能在不忽略应用的情况下来掌握数学理论。再一次在纯数学和应用学科之间建立起有机的结合,在
抽象的共性和色彩缤纷的个性之间建立起牢固的平衡,这或许就是不久的将来数学上的首要任务。

这里不是对数学进行详细的哲学或心理学的分析的地方,但有几点应当强调一下。目前过分强调数学的公理演绎特点的风气,似乎有盛行
起来的危险。事实上,那种创造发明的要素,那种起指导和推动作用的直观要素,虽然常常不能用简单的哲学公式来表述,但是它们却是任何
数学成就的核心,即使在最抽象的领域里也是如此。如果说完善的演绎形式是目标,那么直观和构作至少也是一种动力。有一种观点对科学本身
是严重的威胁,它断言数学不是别的东西,而只是从定义和公理推导出来的一组结论,而这些定义和命题除了必须不矛盾之外,可以由数学家
根据他们的意志随意创造。如果这个说法是正确的话,数学将不会吸引任何有理智的人。它将成为定义、规则和演绎法的游戏,既没有动力也没有
目标。认为灵感能创造出有意义的公理体系的看法,是骗人的似是而非的真理。只有在以达到有机整体为目标的前提下,以及在内在需要的引导下,
自由的思维才能作出有科学价值的成果来。

尽管逻辑分析的思维趋势并不代表全部数学,但它却使我们对数学事实和它们相互间的依赖关系有更深刻的理解,以及对数学中的主要概念有更深刻
的理解,并由此发展了可作为一般科学态度的典范的近代数学观点。

不论我们持什么样的哲学观点,就科学观察的目的来说,对一个对象的人事,完全表现在它与认识者(或仪器)的所有可能关系之中。当然仅仅是
感觉并不能构成知识和见解,必须要与某些基本的实体即“自在之物”相适应、相印证。所谓“自在之物”并不是物体观察的直接对象,而是属于
形而上学的。然而,对于科学方法来说,重要的是应放弃带有形而上学性质的因素,而去考虑那些可观测的事实,把它们作为概念和构作的最终
根源。放弃对“自在之物”的领悟,对“终极真理”的认识以及关于世界的最终本质的阐明,这对于质朴的热诚者来说,可能会带来一种心理上
的痛苦,但事实上它却是近代思想上最有成效的一种转变。